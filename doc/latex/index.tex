\begin{DoxyAuthor}{Autor}
Bartłomiej Olber
\end{DoxyAuthor}
Gra w życie (Life, The game of life) – jeden z pierwszych i najbardziej znanych przykładów automatu komórkowego, wymyślony w roku 1970 przez brytyjskiego matematyka Johna Conwaya.

Gra toczy się na nieskończonej planszy (płaszczyźnie) podzielonej na kwadratowe komórki. Każda komórka ma ośmiu „sąsiadów” (tzw.\+sąsiedztwo Moore’a), czyli komórki przylegające do niej bokami i rogami. Każda komórka może znajdować się w jednym z dwóch stanów\+: może być albo „żywa” (wartość 1), albo „martwa” (wartość 0). Stany komórek zmieniają się w pewnych jednostkach czasu. Stan wszystkich komórek w pewnej jednostce czasu jest używany do obliczenia stanu wszystkich komórek w następnej jednostce. Po obliczeniu wszystkie komórki zmieniają swój stan dokładnie w tym samym momencie. Stan komórki zależy tylko od liczby jej żywych sąsiadów. W grze w życie nie ma graczy w dosłownym tego słowa znaczeniu. Udział człowieka sprowadza się jedynie do ustalenia stanu początkowego komórek.

{\bfseries Reguły gry według Conwaya} \begin{DoxyVerb}Martwa komórka, która ma dokładnie 3 żywych sąsiadów, staje się żywa 
w następnej jednostce czasu (rodzi się).
Żywa komórka z 2 albo 3 żywymi sąsiadami pozostaje nadal żywa; 
przy innej liczbie sąsiadów umiera (z „samotności” albo „zatłoczenia”).
\end{DoxyVerb}


{\bfseries Algorytmy programu}



{\bfseries Opcje programu} \begin{DoxyVerb}F1  Wyjdź z programu
F2  Wyświetl menu opcji użytkownika 
     - Ustaw żywe komórki na planszy
     - Ustaw maksymalną liczbę iteracji
     - Ustaw opóźnienie pomiędzy iteracjami
     - Ustaw rozmiary planszy 
     - Wylosuj stan planszy
F3  Zatrzymaj grę
F4  Strórz nową planszę
F9  Rozpocznij grę   
\end{DoxyVerb}


{\bfseries Analiza pliku wejściowego}

Do analizy pliku wejściowego typu J\+S\+ON została użyta biblioteka Y\+A\+JL.

Oto przykład poprawnego pliku parametrów typu J\+S\+ON

 